\documentclass{article}
\begin{document}
\title{\bfseries Newton's second law}
\maketitle
The second law states that the rate of change of momentum of a body is directly proportional to the force applied, and this change in momentum takes place in the direction of the applied force.
$$\mathbf F=\frac {d \mathbf p} {dt}= \frac {d(m \mathbf v)} {dt}$$
The second law can also be stated in terms of an object's acceleration. Since Newton's second law is valid only for constant-mass systems, {\itshape m} can be taken outside the differentiation operator by the constant factor rule in differentiation.
$$ \mathbf F=m \frac{d \mathbf v} {dt}  $$
Where {\bfseries F} is the net force applied,  {\itshape m} is the mass of the body, and {\bfseries a} is the body's acceleration.
 
\end{document}
	
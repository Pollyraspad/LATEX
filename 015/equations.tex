\documentclass[12pt]{article} 
\usepackage{ucs} 
\usepackage[utf8x]{inputenc} 
\usepackage[russian]{babel}
\usepackage[left=0.2cm,right=0.4cm,top=2cm,bottom=2cm,nohead,nofoot]{geometry}
\linespread{1.7}
\begin{document}

\title{\bfseries 17 уравнений, которые изменили ход истории}
\maketitle 
\begin{tabular}{ 1 1 1 }
\bfseries 1.Теорема Пифагора & $a^2+b^2=c^2$ & Пифагор, 530 год до н.э.  \\
\bfseries 2.Логарифмы & $\log xy=\log x+\log y $ & Джон Непер, 1610 \\
\bfseries 3.Теорема Ньютона-Лейбница& $\frac{df}{dt}=\lim_{h\to 0}=\frac{f(t+h)-f(t)}{h} $ & Ньютон, 1668 \\
\bfseries 4.Закон всемирного тяготения& $F=G \frac{m_1 m_2}{r^2} $ & Ньютон, 1687 \\
\bfseries 5.Мнимая единица & $i^2=-1 $ & Эйлер, 1750\\
\bfseries 6.Теорема Эйлера для многогранников & $V-E+F=2 $ & Эйлер, 1751 \\
\bfseries 7.Нормальное распределение & $ \Phi (x)=\frac{1}{\sqrt{2 \pi \rho}} e^{\frac{(x-\mu)^2}{2 \rho^2}}$  & Гаусс, 1810 \\
\bfseries 8.Волновое уравнение & $\frac{\partial^2 u}{\partial t^2}=c^2 \frac{\partial^2 u}{\partial x^2} $ & Д’Аламбер, 1746 \\
\bfseries 9.Преобразование Фурье & $f(\omega)=\int_\infty^\infty f(x)e^{-2 \pi ix \omega}dx $ & Фурье, 1822 \\
\bfseries 10.Уравнения Навье-Стокса & $ \rho(\frac{\partial \mathbf v}{\partial t}+ \mathbf{v} \cdot \nabla  \mathbf{v}=-\nabla p + \nabla \cdot \mathbf{T}+\mathbf{f}) $ & Навье, Стокс, 1845 \\
\bfseries 11.Уравнения Максвелла & $\nabla \cdot \mathbf{E}=0 $ $\nabla \cdot \mathbf{H}=0 $ & Максвелл, 1865 \\
&$\nabla \times \mathbf{E}=-\frac{1}{c} \frac{\partial \mathbf{H}}{\partial t}$  $\nabla \times \mathbf{H}=\frac{1}{c} \frac{\partial E}{\partial t}$&\\
\bfseries 12.Второй закон термодинамики & $ dS\ge 0 $& Больцман, 1874 \\
\bfseries 13.Теория относительности & $E=mc^2 $ & Эйнштейн, 1905 \\
\bfseries 14.Уравнение Шредингера & $ih \frac{\partial}{\partial t} \Psi=H \Psi$ & Шредингер, 1927 \\
\bfseries 15.Теория информации & $ H=-\sum p(x) \log p(x) $ & Шеннон, 1949 \\
\bfseries 16.Теория хаоса & $x_{t+1}=kx_t(1-x_t) $ & Роберт Мей \\
\bfseries 17.Уравнение Блэка-Шоулза & $\frac{1}{2}\sigma^2S^2\frac{\partial^2V}{\partial S^2}+rS\frac{\partial V}{\partial S}+ \frac{\partial V}{\partial t}-rV=0$ & Блэк, Шоулз, 1990 \\
\end{tabular}


\end{document}

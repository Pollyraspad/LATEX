\documentclass[12pt]{article}
\usepackage[utf8]{inputenc}
\usepackage[russian]{babel}
\usepackage{amsmath}
\usepackage{hyperref} % Cсылки в PDF
\numberwithin{equation}{section} % Изменить нумерацию формул (1) -> (1.1)
\begin{document}
\begin{titlepage}
\begin{center}
\large
МОСКОВСКИЙ ГОСУДАРСТВЕННЫЙ УНИВЕРСИТЕТ \\
имени М.В. ЛОМОНОСОВА \\
МЕХАНИКО-МАТЕМАТИЧЕСКИЙ ФАКУЛЬТЕТ\\ 

\bf А. В. Чашкин \\
\Large 
ЛЕКЦИИ\\
ПО ДИСКРЕТНОЙ МАТЕМАТИКЕ\\
\normalsize
Учебное пособие
\end{center}
\end{titlepage}   
\newpage
\tableofcontents % Вставить содержание
\newpage
\section{Основная теорема}
 Сформулируем и докажем теорему Пойа о сумме весов классов эквивалент-
ности $F$ функций из $D$ в $R$, полагая, что на множестве $D$ действует группа $G$, а на множестве $R$ определена весовая функция $w$ со значениями в коммутативном кольце $K$.
\newtheorem{Th}{Теорема}
\begin{Th}\label{thViet}
Сумма весов классов эквивалентности равна
$$ \sum_F W(F)=P_G \left(\sum_{r\in R}w(r),\sum_{r\in R}(w(r))^2,\dots
,\sum_{r\in R}(w(r))^k\right),$$
где $PG$ — цикловой индекс группы
\end{Th}
{\scshape Доказательство}.Рассмотрим элемент $g$ группы $G$, под действием которого множество $D$ распадается на $k_1$ циклов длины единица, $k_2$ циклов длины два, и т. д. вплоть до $k_s$ циклов длины $s$. Без ограничения общности будем полагать, что циклы длины единица формируются первыми $k1$ элементами множества $D$, циклы длины два формируются следующими $2k_2$ элементами так, что каждый цикл имеет вид $(d_id_{i+1})$, и т. д. Последние $s k_s$ элементов множества
$D$ образуют $k_s$ циклов вида $(d_j d_{j+1} \dots d_{j+s-1})$.
Нетрудно видеть, что вектор значений $v(f)$ любой функции $f$, которая определена на $D$, принимает значения в $R$, и которая под действием элемента $g$ переходит в себя, выглядит следующим образом. На первых $k_1$ местах произвольным образом располагаются любые элементы множества
$R$. Следующие $2k_2$ мест заполнены $k_2$ парами одинаковых элементов из
$R$. Это необходимо и достаточно для выполнения равенства
$f(d) =f(g(d))$ при $d \in \{d_{k_1+1},...,d_{k_1+2k_2}\}$. Следующие $3k_3$ мест заполнены $k_3$ тройками одинаковых элементов из
$R$ и т. д. Наконец последние $s k_s$ разрядов вектора $v(f)$ представляют собой последовательность из $k_s$ блоков длины $s$, каждый из которых состоит из одинаковых элементов. Нетрудно видеть, что все такие векторы можно получить, раскрыв скобки в произведении
\begin{equation}
\left(\sum_{r \in R}r\right)^{k_1}
\left(\sum_{r \in R}r r\right)^{k_2}
\dots
\left(\sum_{r \in R} \underbrace{r \dots r}_s\right)^{k_s},
\label{eq.1}
\end{equation}
полагая при этом, что умножение в~\eqref{eq.1} некоммутативно. Таким образом,
\begin{equation}
\sum_{f=g(f)}v(f)=
\left(\sum_{r \in R}r\right)^{k_1}
\left(\sum_{r \in R}r r\right)^{k_2}
\dots
\left(\sum_{r \in R} \underbrace{r \dots r}_s\right)^{k_s},
\label{eq.2}
\end{equation}
Например, если $s=2$, $k_1=1$, $k_2=2$ и $R=\{x,y\}$, то
\begin{eqnarray}
(x+y)(xx+yy)(xx+yy)=x\ xx \ xx + x\ xx\ yy + x\ yy\ xx + 
\nonumber  \\
 + x\ yy\ yy + y\ xx\ xx + y\ xx\ yy + y\ yy\ xx + y\ yy\ yy.
\label{eq.3}
\end{eqnarray}
Теперь вычислим сумму весов всех функций, которые под действием
элемента $g$ переходят в себя. Для этого в~\eqref{eq.2} заменим каждый элемент $r$ его весом $w(r)$. Тогда, в силумультипликативности функции $w$,
\begin{equation}
\sum_{f=g(f)}w(v(f))=
\left(\sum_{r \in R}w(r)\right)^{k_1}
\left(\sum_{r \in R}(w(r))^2\right)^{k_2}
\dots
\left(\sum_{r \in R}(w(r))^s\right)^{k_s}.
\label{eq.4}    
\end{equation}
Допустим, что веса функций, оставляемых элементом $g$ на месте, прини-
мают значения $w_1, \dots ,w_m$. Тогда сумму весов рассматриваемых функций можно представить в виде
\begin{equation}
\sum_{w_i}w_i \psi_i (g)
\label{eq.5}     
\end{equation}
где $\psi_i(g)$ — число функций веса $w_i$. Сумма именно такого вида получится после открытия скобок в правой части равенства~\eqref{eq.4} и последующего приведения подобных слагаемых. Продолжая рассмотренный выше пример, положим $w(x)=t$, $w(y)=s$ и вычислим сумму весов всех функций, векторы значений которых перечислены в правой части равенства~\eqref{eq.3}. Нетрудно видеть, что
\begin{eqnarray}
(w(x) +w(y))(w(xx) +w(yy))(w(xx) +w(yy)) =
\nonumber  \\
= (t+s)(t^2+s^2)(t^2+ s2)=t^5+t^4s+2t^3s^2+2t^2s^3 +ts^4+s^5,
\nonumber
\end{eqnarray}
где коэффициент при одночлене $t^i s^j$ равен количествутех функций, вескоторых равен $t^is^j$.
Возвращаясь к равенству~\eqref{eq.4} , заметим, что произведение в его правой части есть ничто иное, как индекс элемента $g$, в который вместо переменных $z_k$ подставлены суммы $\sum_{r \in R}(w(r))^2$. Следовательно, сумма весов всех функций, которые под действием элемента $g$ переходят в себя, равна
\begin{equation}
I_g \left(\sum_{r \in R}w(r),(\sum_{r \in R}(w(r))^2, \dots, (\sum_{r \in R}(w(r))^s \right).
\label{eq.6}
\end{equation}
Вычислив сумму величин~\eqref{eq.6} по всем элементам группы $G$ и разделиврезультат на порядок группы $G$, видим, что в силу~\eqref{eq.5}
\begin{eqnarray}
P_G \left(\sum_{r \in R}w(r),(\sum_{r \in R}(w(r))^2, \dots, (\sum_{r \in R}(w(r))^k, \dots \right)= 
 \nonumber  \\
= \frac{1}{|G|}\sum_{g \in G}\sum_{w_i}w_i \psi_i(g)
= \sum_{w_i}w_i
\left(\frac{1}{|G|}\sum_{g \in G} \psi_i(g) \right).
\nonumber 
\end{eqnarray}
Из леммы Бернсайда следует, что при фиксированном значении веса $w$ сумма $\frac{1}{|G|}\sum_{g \in G} \psi_i(g)$  равна числуклассов эквивалентности, возникающих на
множестве функций веса $w_i$ в результате действия группы $G$ на множестве $D$. Следовательно, левая часть последнего равенства равна сумме весов всех классов эквивалентности. Теорема доказана.
\end{document}